\documentclass{article}
\usepackage[utf8]{inputenc}
\usepackage{amsmath}
\usepackage{amssymb}
\usepackage{listings}
\usepackage{hyperref}
\newcommand{\abs}[1]{\left|#1\right|}
\newcommand{\floor}[1]{\left\lfloor#1\right\rfloor}
\lstset{
  columns=flexible,
  basicstyle={\small\ttfamily}
}
\usepackage[margin=3cm]{geometry}
\setcounter{secnumdepth}{0}
\begin{document}

\section{Ořezávání účtenky ve fotografii}
Specifikace ročníkového projektu\\
25.2.2019 Jiří Mayer

\paragraph{Cíl projektu\\}

Vytvořit program, který dostane fotografii účtenky a vrátí vrcholy čtyřúhelníku,
který co nejlépe ohraničuje účtenku. Oříznutí podle čtyřúhelníku by mělo
účtenku připravit na OCR.

\paragraph{Vlastnosti vstupu\\}

Fotografie přímo z chytrého telefonu. Fotografie je ostrá a má
dostatečné rozlišení. Je vyfocená na výšku. Před zpracováním
se její šířka normalizuje na nějakou danou velikost.

\paragraph{Očekávaná přesnost\\}

Program by měl detekovat fotografie:

\begin{itemize}
    \itemsep -0.3em
    \item s jednotnou texturou na pozadí (dřevo, kov, plast, látka)
    \item pokud jsou na fotografii kromě účtenky jiné předměty nebo okraj stolu
    \item pokud účtenku někdo přidržuje prsty nebo je zatížená nějakým předmětem
    \item osvětlení by nemělo hrát roli (den, noc, svítilna na mobilu)
    \item část účtenky je mimo okraj fotografie (malá část, hlavně nahoře nebo dole)
\end{itemize}

Situace, které nejspíš nebudou dobře detekovány:

\begin{itemize}
    \itemsep -0.3em
    \item pozadí má podobnou texturu, jako účtenka (noviny, jiné účtenky, bílá plocha)
    \item účtenka je moc zmačkaná a tak má špatný tvar nebo texturu
    \item přes účtenku leží předměty tak, že její tvar je moc zvláštní nebo dokonce nesouvislý
\end{itemize}

Pokud účtenka není nalezena, program informuje uživatele.

\paragraph{Použité technologie\\}

Program bude realizovaný jako konzolová aplikace v Pythonu s použitím
knihoven OpenCV a Numpy.

\paragraph{Představa implementace\\}

Začnu podobným postupem, jaký je uvedený v článku Donoser, Bischof,
Wiltsche \url{https://www.researchgate.net/publication/224057460_Color_blob_segmentation_by_MSER_analysis} a zkusím místo
explicitního vybírání počáteční oblasti poskytnout nějakou sadu
dobrých distribucí pixelů. Nebo můžu kombinovat obojí. Použiju MSER na
detekci blobu účtenky, případně jejího konvexního obalu.

Pro detekci čtyřúhelníku z blobu nechci používat bounding rectangle,
doufám že najdu lepší algoritmus, který se bude snažit trefit okraje účtenky
a ne minimalizovat obsah oblasti.

Jelikož mám problém v dost úzké doméně, tak si můžu dovolit předpokládat
víc věcí (a tak například určit některé parametry na základě dat).

\qquad

Předpokládám, že konkrétní implementace se ještě v průběhu změní nebo upraví,
protože teď nedokážu říct, zda bude zvolený postup splňovat všechny požadavky.

\paragraph{Rozšíření\\}

Pokud by program nebyl dostatečně rozsáhlý, můžu přidat buď další krok
v OCR pipeline a sice binarizaci fotografie, nebo můžu implementovat
jiný detektor a výsledek nějak průměrovat z obou (více) detektorů.

\end{document}